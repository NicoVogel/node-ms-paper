%!TEX root = ../../dokumentation.tex


There are different types and sizes for \ac{LAN} parties.
The purpose of the \ac{LAN} party management application is to manage multiple \ac{LAN} parties over time.
The general idea is that users can create an account, sign up for an upcoming \ac{LAN} party (or event) and participate with their account.
Consequently, the application could contain the following components:

\paragraph{Accounts} are used to identify a user.
Multiple accounts can be grouped to form a team in an event.
This is required for some games.

\paragraph{Events} represent actual \ac{LAN} parties.
They consist of information such as start/end date and a description.
This information could be used for the registration page of an event.
Events could be required to pay a registration fee, which is also part of this component.
Finally, a form of attendee overview could be included.

\paragraph{Scoreboard} is responsible for all matches.
This component therefore contains the following information: participants, played game, used settings and the match result.
It could also be used to manage the game server of the event and to start matches.

\paragraph{Catering} could be part of an event.
Therefore, a list of catering options within the event registration could be used to manage catering.
The result would be an overview of the catering options, chosen by attendees.
This would make it easier to order an actual catering service.

\paragraph{Billing} is an essential part of an event.
A registration is only complete once the registration fee and catering costs have been paid.
A form of billing is therefore required.

\paragraph{Seating plan} option would automate the process of seat selection.
A participant could select a seat in the event room and reserve it in a similar way to reserving a seat in a cinema.

These components/features are used as a general guide for creating the \ac{LAN} party application.
Not every feature is required for every event.
For example, an event might not provide an option for catering.

Lastly, the \ac{LAN} party application architecture will be client-server based, where the client is a web browser.
This modern approach allows for a platform independent use.