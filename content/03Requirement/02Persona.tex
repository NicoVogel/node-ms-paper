%!TEX root = ../../dokumentation.tex

There are many combinations of technologies to implement an application based on microservices, but not every combination meets the requirements for such an application.
Technologies must meet the requirements to be useful for an application.
A common practice in web development is to define personas, which have requirements for an application \cite[p. 105]{Castro.2008}.
A persona reflects the interests of a certain group of people.
By defining several personas, several aspects for an application are considered.
Personas have been successfully used in non-Web development projects \cite[P. 105]{Castro.2008} and are used in requirement analysis.

In case of technologies comparison, suitable personas are \textbf{end-user}, \textbf{developer} and \textbf{operator}.
These personas are selected based on the writer's knowledge.
For an end-user, there are many factors that significantly influence the acceptance of the application \cite[p. 245]{Zhou.2008}, but only the performance expectancy can be influenced by communication technologies.
The performance expectancy is important to determine interest of the end-user in the application \cite[p. 244]{Zhou.2008}.
From the end-user's point of view, the application should be performant, so the communication technology must be performant as well.

Developers are primarily interested in whether a communication technology contains all features they need to perform a task.
Sometimes the customer decides which technology to use.
This is the case when a developer is part of a consultant firm and the client has heard of a technology that he wants.
In this case, sometimes a technology is chosen because the customer understands it or is used to it.
When developers have a choice of which technology to use and several technologies contain all the necessary features, they evaluate which technology fits the scenario and meets the performance and scalability requirements the most.
Finally, they consider usability features such as implementation effort and manageability.
Therefore, from a developer's perspective, a communication technology is selected based on performance, scalability, functional requirements and ease of use \cite{ManagingSolutionArchitect.20.05.2020}.

The last persona is the operator.
For this persona, scaling and managing complexity of communication technology is most important.
Features like historization and versioning are a must have and depending on the project, the communication flow must be configurable.
Just like the developer, the operator points out that the customer plays a major role in the choice of communication technology and there is usually little or no choice. Another concern is resilience, which is also a key factor for microservices.
Finally, non-functional requirements such as monitoring and traceability are also important.
Therefore, from an operator perspective, a communication technology is selected based on scalability, resilience and the inclusion of some standard features \cite{EnterpriseArchitectforcloudbasedinfrastructure.20.05.2020}.

After explaining the different needs of personas, Table \ref{tab:requirementsAnalysis} provides an overview of the important factors for each persona. It is used for the technology assessment.
Further information about the interviews for the developer and operator can be found in the appendix at \ref{chp:appendix:interview}.

\begin{table}
	\centering
	\begin{tabular}{ |c|c|c| }
		\hline
		End-User    & Developer   & Operator    \\
		\hline
		Performance & Performance & Scalability \\
		            & Scalability & Resilience  \\
		            & Ease of use &             \\
		\hline
	\end{tabular}
	\caption{Requirement Analysis Overview} \label{tab:requirementsAnalysis}
\end{table}