%!TEX root = ../dokumentation.tex

\pagestyle{empty}

\iflang{de}{%
	% Dieser deutsche Teil wird nur angezeigt, wenn die Sprache auf Deutsch eingestellt ist.
	\renewcommand{\abstractname}{\langabstract} % Text für Überschrift

	% \begin{otherlanguage}{english} % auskommentieren, wenn Abstract auf Deutsch sein soll
	\begin{abstract}

	\end{abstract}
	% \end{otherlanguage} % auskommentieren, wenn Abstract auf Deutsch sein soll
}



\iflang{en}{%
	% Dieser englische Teil wird nur angezeigt, wenn die Sprache auf Englisch eingestellt ist.
	\renewcommand{\abstractname}{\langabstract} % Text für Überschrift

	\begin{abstract}

		This work focuses on the selection of microservice technologies in the field of communication and monitoring.
		For the selection, persona requirements are identified and excerpts of currently available technologies presented for a preselection.
		The selection was guided by the use case of a management application for gatherings of people (\enquote{LAN party}).
		Following the preselection, proof-of-concepts were made for a better evaluation (except for the aspect monitoring), which resulted in the final selection being \textit{RabbitMQ (AMQP)} for asynchronous, \textit{RESTful HTTP} for synchronous communication and lastly \textit{Prometheus} for monitoring.
		The exemplary implementation of said management application proved the interoperability and, albeit not having covered other aspects in a microservice environment, may serve as a guideline for future projects.

		Further technologies presented were: \textit{gRPC, GraphQL (prototyped), NATS, Apache Kafka (prototyped), Graphite}
	\end{abstract}
}