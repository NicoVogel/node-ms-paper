%!TEX root = ../dokumentation.tex

\addchap{\langabkverz}
%nur verwendete Akronyme werden letztlich im Abkürzungsverzeichnis des Dokuments angezeigt
%Verwendung: 
%		\ac{Abk.} --> fügt die Abkürzung ein, beim ersten Aufruf wird zusätzlich automatisch die ausgeschriebene Version davor eingefügt bzw. in einer Fußnote (hierfür muss in header.tex \usepackage[printonlyused,footnote]{acronym} stehen) dargestellt
%		\acs{Abk.} --> fügt die Abkürzung ein
%		\acf{Abk.} --> fügt die Abkürzung UND die Erklärung ein
%		\acl{Abk.} --> fügt nur die Erklärung ein
%		\acp{Abk.} --> gibt Plural aus (angefügtes 's'); das zusätzliche 'p' funktioniert auch bei obigen Befehlen
%	siehe auch: http://golatex.de/wiki/%5Cacronym
%	
\begin{acronym}[YTMMM]
	\setlength{\itemsep}{-\parsep}

	\acro{AMQP}{Advanced Message Queuing Protocol}
	\acro{API}{Application Programming Interface}
	\acro{HTML}{Hypertext Markup Language}
	\acro{HTTP}{Hypertext Tranfer Protocol}
	\acro{I/O}{Input Output}
	\acro{IDE}{Integrated Development Environment}
	\acro{IDL}{Interface Definition Language}
	\acro{IoT}{Internet of Things}
	\acro{JSON}{JavaScript Object Notation}
	\acro{LAN}{Local Area Network}
	\acro{MQTT}{Message Queuing Telemetry Transport}
	\acro{NATS}{Neural Autonomic Transport System}
	\acro{QoS}{Quality of Service}
	\acro{RAM}{Random Access Memory}
	\acro{REST}{Representational State Transfer}
	\acro{RPC}{Remote Procedure Call}
	\acro{SQL}{Structured Query Language}
	\acro{STOMP}{Simple Text Orientated Messaging Protocol}
	\acro{TLS}{Transport Layer Security}
	\acro{TSDB}{Time Series Database}
	\acro{UI}{User Interface}
	\acro{URL}{Uniform Resource Locator}
	\acro{XML}{Extended Markup Language}
\end{acronym}
